% This must be in the first 5 lines to tell arXiv to use pdfLaTeX, which is strongly recommended.
\pdfoutput=1
% In particular, the hyperref package requires pdfLaTeX in order to break URLs across lines.

\documentclass[11pt]{article}

% Change "review" to "final" to generate the final (sometimes called camera-ready) version.
% Change to "preprint" to generate a non-anonymous version with page numbers.
\usepackage[preprint]{acl}

% Standard package includes
\usepackage{times}
\usepackage{latexsym}

% For proper rendering and hyphenation of words containing Latin characters (including in bib files)
\usepackage[T1]{fontenc}
% For Vietnamese characters
% \usepackage[T5]{fontenc}
% See https://www.latex-project.org/help/documentation/encguide.pdf for other character sets

% This assumes your files are encoded as UTF8
\usepackage[utf8]{inputenc}

% This is not strictly necessary, and may be commented out,
% but it will improve the layout of the manuscript,
% and will typically save some space.
\usepackage{microtype}

% This is also not strictly necessary, and may be commented out.
% However, it will improve the aesthetics of text in
% the typewriter font.
\usepackage{inconsolata}

%Including images in your LaTeX document requires adding
%additional package(s)
\usepackage{graphicx}

% If the title and author information does not fit in the area allocated, uncomment the following
%
%\setlength\titlebox{<dim>}
%
% and set <dim> to something 5cm or larger.

\title{Approaches to automatic detection of machine-generated text}

% Author information can be set in various styles:
% For several authors from the same institution:
% \author{Author 1 \and ... \and Author n \\
%         Address line \\ ... \\ Address line}
% if the names do not fit well on one line use
%         Author 1 \\ {\bf Author 2} \\ ... \\ {\bf Author n} \\
% For authors from different institutions:
% \author{Author 1 \\ Address line \\  ... \\ Address line
%         \And  ... \And
%         Author n \\ Address line \\ ... \\ Address line}
% To start a separate ``row'' of authors use \AND, as in
% \author{Author 1 \\ Address line \\  ... \\ Address line
%         \AND
%         Author 2 \\ Address line \\ ... \\ Address line \And
%         Author 3 \\ Address line \\ ... \\ Address line}

\author{Aron Winkler \\
  University of Tuebingen \\
  MAT 6189673 \\
  \texttt{aron.winkler@student.uni-tuebingen.de} \\}

%\author{
%  \textbf{First Author\textsuperscript{1}},
%  \textbf{Second Author\textsuperscript{1,2}},
%  \textbf{Third T. Author\textsuperscript{1}},
%  \textbf{Fourth Author\textsuperscript{1}},
%\\
%  \textbf{Fifth Author\textsuperscript{1,2}},
%  \textbf{Sixth Author\textsuperscript{1}},
%  \textbf{Seventh Author\textsuperscript{1}},
%  \textbf{Eighth Author \textsuperscript{1,2,3,4}},
%\\
%  \textbf{Ninth Author\textsuperscript{1}},
%  \textbf{Tenth Author\textsuperscript{1}},
%  \textbf{Eleventh E. Author\textsuperscript{1,2,3,4,5}},
%  \textbf{Twelfth Author\textsuperscript{1}},
%\\
%  \textbf{Thirteenth Author\textsuperscript{3}},
%  \textbf{Fourteenth F. Author\textsuperscript{2,4}},
%  \textbf{Fifteenth Author\textsuperscript{1}},
%  \textbf{Sixteenth Author\textsuperscript{1}},
%\\
%  \textbf{Seventeenth S. Author\textsuperscript{4,5}},
%  \textbf{Eighteenth Author\textsuperscript{3,4}},
%  \textbf{Nineteenth N. Author\textsuperscript{2,5}},
%  \textbf{Twentieth Author\textsuperscript{1}}
%\\
%\\
%  \textsuperscript{1}Affiliation 1,
%  \textsuperscript{2}Affiliation 2,
%  \textsuperscript{3}Affiliation 3,
%  \textsuperscript{4}Affiliation 4,
%  \textsuperscript{5}Affiliation 5
%\\
%  \small{
%    \textbf{Correspondence:} \href{mailto:email@domain}{email@domain}
%  }
%}

\begin{document}
\maketitle

\section*{Abstract}

Recent developments in Natural Language Processing (NLP) have resulted in the development and popularization of highly effective Large Language Models (LLMs), capable of generating convincing and seemingly creative linguistic material.
LLMs have garnered much attention, both from researchers and the general public, and continue to be increasingly applied to a variety of fields. However, the breakneck speed at which these systems are adopted leaves unattended some of the security concerns regarding their use.
Since the language produced by the models is of such high quality, it is not always feasible to distinguish authentically human contributions from machine-generated text, which can enable a multitude of nefarious applications of LLMs.
This work explores the history and inner workings of LLMs, how they can be misused, and possible antidotes to the problem of machine-generated text detection, with a careful eye toward a good balance of computational cost and performance of detection strategies.

\section{Introduction}

In 1951, Gnome Press published Isaac Asimov's \emph{Foundation} \citep{asimov1951foundation}, the first title of a trilogy that would go on to become one of the cornerstones of modern science fiction. In the novel, set in the distant future, scientist Hari Seldon predicts the fall of the Galactic Empire, an event that would pave the way to an era of barbarism in the story's fantastical universe.

To preserve humanity's knowledge and technical skills, Hari Seldon establishes the Foundation on an uninhabited planet on the periphery of the Empire, a sort of outpost dedicated to being the home to the archival effort. The novel follows the political and technological adventures of the Foundation and its leaders, with one of the first plot points being the first conflict between the Foundation and a major local power in the periphery, Anacreon, which declared its independence as the Empire's influence in the periphery weakened. Seeking protection from the Empire against Anacreon's expansionary stance, the Foundation hosts a diplomatic emissary from the Empire, a Lord Dorwin, finally obtaining a convoluted treaty between the Empire and Anacreon over their respective spheres of influence.

\begin{quote}
    "Before you now you see a copy of the treaty between the Empire and Anacreon – a treaty, incidentally, which is signed on the Emperor's behalf by the same Lord Dorwin who was here last week – and with it a symbolic analysis."

    The treaty ran through five pages of fine print and the analysis was scrawled out in just under half a page.

    "As you see, gentlemen, something like ninety percent of the treaty boiled right out of the analysis as being meaningless, and what we end up with can be described in the following interesting manner:

    "Obligations of Anacreon to the Empire: None!"

    "Powers of the Empire over Anacreon: None!"

    \vspace{0.2cm}

    \begin{flushright}
        \small \emph{Isaac Asimov, Foundation,\\Part II: The Encyclopedists}
    \end{flushright}
\end{quote}

At this point the Foundation's scientists, through a technique they call \emph{"symbolic analysis"}, condense several pages of treaty into a few lines, revealing the hidden meaning behind the layers of legal dissimulation. By doing so, they expose the inability of the dying empire to exert its influence over its own periphery, and they realize that moving forward, they can only rely on themselves, marking perhaps the true starting point of the story in Asimov's \emph{Foundation}.

Despite first reading this passage when I was a teenager, perhaps over a decade ago, these fictional twists stayed with me through the years. They were, after all, my first indirect exposure to the field of computational linguistics and natural language processing (NLP). I remember being mesmerized by the potential of machine computation applied to natural language, in what I would later learn to better define as a mixture of information retrieval and automatic text summarization.

While Asimov's pen definitely hit the mark in predicting some of the most intriguing and successful applications of NLP in the years ahead, what granted computational linguistics perhaps its brightest moment in the limelight was one of its other, albeit related, subfields: language generation and modelling.

\subsection{Language modelling}

Teaching a machine to understand and produce natural language is intuitively a difficult task. Even if one could reliably collect all ingredients that make up human language, creating a system that emulates it even just well-enough is a very tall order, since there would likely be millions if not billions of cases to consider. Linguists have documented hundreds of languages, each with their own grammar, peculiarities, exceptions, all of which have yet to be described under one common ruleset. Manually building a program from the ground up for even just one language is beyond what current technology is capable of.

The very first chatbot, ELIZA \citep{citationneeded}, simulated conversation through pattern matching and substitution, essentially repeating and paraphrasing their interlocutor's statements. While it successfully bypasses the necessity of programming a machine with \emph{intelligence}, such an approach does not result in a system that can be described as creative in any sense. In other words, ELIZA will never write a poem, or surprise their conversation partner with a witty turn of phrase. It would never be able to tell whether May has 30 or 31 days because it has no notion of what \emph{May} and \emph{days} are. Teaching language is, after all, not only an issue of grammar, but one of world knowledge as well.

If \emph{teaching} language to machines as one would to humans is not possible, and rule-based approaches such as ELIZA inevitably reach a bottleneck, then it becomes necessary to adopt a new strategy, rooted in statistics. This new approach consists in the realization that the sentence "he's wearing a circumference jacket" is much less likely to be uttered than "he's wearing a yellow jacket". Extrapolating the pattern, the set of words that can fill the gap in "he's wearing a \_\_\_\_\_ jacket" is varied, but "yellow" will have a much higher \emph{probability} of showing up than "circumference". Language models are the tools that are employed to estimate these probabilities.

Due to recent innovations in NLP, the phrase "language model" evokes big and expensive systems, trained on huge amounts of data and costing enormous amounts of money to develop. While this is certainly understandable, the label in itself has no presupposition of size or cost. In essence, language models break down the massively complex problem of "teaching language to machines", into the more manageable task of "statistically learning what words are likely to follow others". In other words, language models produce next-word (or, more generally, next-token) probabilities based on an input sequence \citep{citationneeded}.

For example, for the completion "fifteen minutes of \_\_\_\_\_", one would expect a (good) language model to offer words such as "fame" or "overtime". One idea to achieve this is to collect some linguistic data and observe what words follow "fifteen minutes of" and extrapolate a probability distribution from the observed frequencies. So-called \emph{n-gram} language models \citep{citationneeded} are built in this fashion, with the \emph{n} in \emph{n-gram} specifying the amount of left context taken into consideration.

The simplest of these models, the bigram (2-gram) language model, records co-occurring word pairs in the sample dataset. This means that for this model, only the last word of a sequence determines the prediction over the following word. This results in a model that can reliably generate short collocations, such as "Marie \emph{Curie}", but cannot generate coherent sentences, and would likely even fail to offer "fame" as a completion to "fifteen minutes of \_\_\_\_\_", since the only available context for the prediction is the word "of". To correctly predict "fame", one would need at least a 4-gram language model, which would finally allow for such a "long" context requirement. However, while taking more context tokens into consideration increases the performance of n-gram language models, it does so at a steep (especially memory) cost: for 4-gram LMs with a vocabulary size (i.e., how many words the model knows) of 1000, for example, implementations without optimizations would require the frequency counts for $10^{4}$ n-grams to be accessible for predictions.

% \emph{"Just ask chatGPT"} is a commonplace phrase nowadays, sometimes uttered even with some contempt (more on that later). The intuition behind \emph{"Just ask chatGPT"} is, however, exactly what makes language models such a powerful and malleable tool in the hands of computational linguists.

% As such, the simple question of language modelling is in itself quite simple: given a language sequence, what are the likely following entries?

\subsection{Large Language Models}
\subsection{Fears and reactions to LLMs}
\section{Dangers of undetected generation}
\section{Previous approaches}
\label{sec:approaches}

If the above foray into the threat models concerning language generation that have been discussed in research has revealed anything, it is that expecting model operators to disclose the nature of texts submitted to their recipients is a naive and potentially dangerous habit.
Thankfully, as language models have grown larger and more sophisticated, so have strategies and technologies detecting them matured alongside them.
Though it has certainly received and influx of manpower interest since then, the field of automatic detection of machine generated text has been an active area of research even from before 2020 -- after all, as was previously discussed, there was no shortage of attempts to exploit NLG technologies well before the recent GPT craze.

In Chapter \ref{sec:background}, this thesis presented an overview of different ways in which text could be represented, on a spectrum ranging from surface-level metrics to neural contextual embeddings.
Many parallels can be drawn between these strategies and the methodologies employed to detect machine generation.
In the following sections, several well researched and empirically tested detection strategies will be explored, some taking advantage of linguistically motivated features or even frequency-based metrics, while others lean into the times and use LLMs themselves to discriminate between authentic and generated productions.
It should however be underlined that the movement toward more sophisticated vectorization processes does not necessarily entail a one-sided improvement across all possible aspects involved in the detection pipeline.

On the contrary, moving up the ladder of complexity comes at the cost of requiring increasing amounts of computational power.
Such approaches are extremely valid explorations of what is the best that can be achieved, but applicability in the real world is often limited since the end users cannot run the required software on commonplace commercial machines.
Separating the application into client and server, which is the common approach taken in other AI fields, is only acceptable in the narrower subset of cases which have no privacy requirement regarding the data being verified.
Chapter \ref{sec:task} will dive deeper into the more "cheap" approaches, meaning those that forgo heavy systems, aiming instead to strike a balance between performance and (compute) accessibility.

\subsection{Frequency and feature-based methods}

A simple way to think of documents is as so-called bags of words (BOW) \citep{murphy2006naive}.
BOW representation strategies only consider the words that appear in a text, without maintaining any information about the order.
One application is the simple count vectorization strategy \citep{wendland2021introduction}, in which a text is represented by the counts of each word that occurs in it.

A closely related, but more refined and widely applied method is the extraction of TF-IDF matrices over sets of documents \citep{ramos2003using}.
If one only looks at word counts, then prepositions and other common elements will likely account for much of the information retrieved from the text.
TF-IDF offers a fix by implementing the intuition that a text containing the words "geothermal" and "power-plant", even if only with a few occurrences, is more informative about its nature than it having hundreds of examples "the" and "of".
As such, term frequencies in TF-IDF are weighted by how rare they are (i.e. in how many documents of the sample they appear), with rarer words gaining a proportional score boost.

TF-IDF has a rich history of successful usage in almost all fields of NLP, including information retrieval \citep{ramos2003using}, sentiment analysis \citep{cahyani2021performance}, and text-classification \citep{zhang2011comparative} among many others.
Machine text detection has also tapped into this tradition. 
Recently, \citet{frohling2021feature} employed higher-dimensional TF-IDF both as a classification accuracy baseline and as an input to meta-learners, i.e. ensemble models that take predictions from multiple classifiers to produce a final label.
While falling short of state-of-the-art solutions that will be discussed later in this chapter, this methodical experiment still managed to build competitive systems by combining (computationally) simple approaches, in part by drawing upon the information captured by TF-IDF.

Alongside n-gram frequency features, like the ones discussed so far, another common approach to text vectorization is represented by linguistic feature sets.
Instead of relying on simple token counts to derive a way to represent human productions with numbers, this strategy utilizes the quantifiable properties of the language being used.
Some trivial examples may be the global length of a text, the average word length of the sentences contained within it, the type-token ratio, and so forth.
Naturally, one can devise much more refined metrics to extract from texts, for example regarding text fluency, readability, grammatical properties, the richness of the vocabulary, cohesion, or even its purpose.
The inquiry mentioned above, conducted by \citet{frohling2021feature}, combined TF-IDF representations with feature-based classifiers across a variety of settings and datasets -- in fact, these linguistic features drive most of the best models' performance.

The first targetable linguistic characteristic that language models have been observed to exhibit is a lack of cohesion \citep{holtzman2019curious}. 
A study on GPT-2, for example, found that decoding strategies that maximize overall probability are likely to run into repetitive language \citep{see2019massively}, and even topic-drift, a phenomenon in which language models fail to stay within the confines of a particular topic \citep{badaskar2008identifying}.
Traditional readability features, such as the Gunning-Fog Index and the Flesch-Kincaid Index, have been used successfully to identify generated text \citep{Crothers_2022}.
Other methods involve modeling relationships between entities across documents through auxiliary models \citep{barzilay2008modeling}, which seek to automatically determine the primary elements of the text, and track how they are presented throughout a text -- with the idea being that human authors will refer back to the main points more often than machines.

Another useful set of features involve the use of varied vocabulary, and one that avoids repetition as much as possible.
Texts authored by humans typically exhibit creative strategies to maintain narrative flow, such as building deep coreference chains instead of bogging down the text with frequent repetitions \citep{feng2010comparison} -- LMs tend instead to err toward toward the opposite side \citep{gehrmann2019gltrstatisticaldetectionvisualization}.
Generated text display frequent use of repetition, leading to reduced lexical diversity within the text \citep{zellers2020evaluating}.
Lexical richness is, fortunately, one of the language aspects that can be more readily measured through features such as type-token ration, content or stop-word ratio, POS-tag distribution, and frequency of rare words, among many \citep{van2007comparing}.
\citet{see2019massively} identify a concrete trend in word-type usage in generated texts, where LMs favor verbs and pronouns more than humans, who make richer use of nouns and adjectives.
Moreover, the reduced usage of varied ways to refer to entities can be measured by extracting the length of coreference chains \citep{feng2010comparison}, with the expectation that machines will produce shorter chains and more explicit repetitions.

The repetitive nature of generated text can also be attributed to the underlying sampling strategies.
\citet{ippolito2019automatic} observe that a huge proportion of the probability mass is concentrated in the first few hundred, most common tokens in the case of top-k sampling.
To make use of this property, another set of features measuring frequency of common texts can be employed, and it can lead to good detection performance when the model is sampled with top-k (other sampling methods do not necessarily exhibit the same tendency).


\begin{enumerate}
    \item https://arxiv.org/abs/1904.09751
    \item https://peerj.com/articles/cs-443/
    \item https://ieeexplore.ieee.org/abstract/document/9892269
    \item https://ieeexplore.ieee.org/abstract/document/8282270
    \item https://arxiv.org/abs/2111.02878
\end{enumerate}

\subsection{Neural approaches}

\begin{enumerate}
    \item Adversarial Robustness of Neural-Statistical Features in Detection of Generative Transformers
    \item Defending against neural fake news
    \item Cross-Domain Detection of GPT-2-Generated Technical Text
    \item Real or Fake? Learning to Discriminate Machine from Human Generated Text
    \item DetectGPT: Zero-Shot Machine-Generated Text Detection using Probability Curvature
\end{enumerate}


\subsection{Domain differentiation}

\begin{enumerate}
    \item Technical: Cross-Domain Detection of GPT-2-Generated Technical Text
    \item Socials:  Automatic Detection of Bot-Generated Tweets
    \item Socials, reviews: Detecting computer-generated disinformation
    \item Socials: Deep Fake Recognition in Tweets Using Text Augmentation, Word Embeddings and Deep Learning
    \item Chatbots:  Detecting Bot-Generated Text by Characterizing Linguistic Accommodation in Human-Bot Interactions
    \item Ecom: Creating and detecting fake reviews of online products.
\end{enumerate}
\section{Task 8 at SemEval 2024}
\label{sec:task}

The 18th International Workshop on Semantic Evaluation (SemEval-2024) was a large event in NLP, offering various challenges that teams across the world could undertake.
Task 8 at SemEval-2024 \citep{wang2024semeval} centered around machine-generated text detection in a black-box setting (i.e., the generator models for the test set were not known while the competition was ongoing), in both monolingual and multilingual components.

This shared task spanned 3 subtasks: subtask A a binary classification task between generated and human texts, subtask B consisted in multi-class classification between multiple LLM generators, as well human texts, while subtask C was a boundary detection problem, where participants had to correctly pinpoint the boundary between a human and a machine-generated segment in each test.
During the active phase of SemEval-2024, which ended in February 2024, I undertook Task 8 jointly with Daniel Stuhlinger, a fellow student at the University of Tübingen.
The full report of our participation, carried out under the team name "TueCICL", can be viewed in \citet{stuhlinger-winkler-2024-tuecicl}.

As team TueCICL, we submitted results for two of the subtasks to the leaderboards, namely subtask A and subtask C, which were the two subtasks involving classification between solely human and machine-generated texts, whereas subtask B focused more on inter-generator differentiation.
For subtask A, consisting in binary classification over the whole texts, there were two phases development, one for shared task proper, and subsequent post-deadline experimenting in the context of this Master Thesis.
Due to the more pronounced research interest in subtask A, as well as to better present the higher volume of material associated with it, this section shall first discuss subtask C, then dive deeper into subtask A later in this Chapter.

\subsection{Human - LM Boundary detection}

Subtask C of the shared task addresses detection environments that are characterized by active adversarial agents performing the generation.
Specifically, detection technologies have been observed to be weak to techniques also found in obfuscation \citep{macko2024authorship}, such as paraphrasing \citep{krishna2024paraphrasing} and noise-introduction \citep{wang2021adversarial}.
Change point detection, which is the titular requirement of subtask C, addresses another way in which the use of NLG technology might be obfuscated.
In the scenario targeted by the subtask, a human segment ranging from 0\% to 50\% of the text is concluded by a machine-generated component, making the text as a whole much harder to flag as a generation.

Task organizers provided generations by GPT variants and the LLaMA series \citep{touvron2023llama}, but unfortunately not in high abundance: the training set around 5000 texts, and was accompanied by development set containing a little over 500 more.
Table \ref{fig:taskc_data} offers an overview of the data distribution across the sets.


\begin{figure}[h]
    \centering
    \includegraphics[width=0.5\textwidth]{assets/subtaskc-data.png}
    \caption{
        Dataset breakdown for subtask C from Task 8 at SemEval-2024.
        The number in “()” is the number of examples purely generated by LLMs, i.e., human and machine boundary index=0.
        LLaMA-2-7B* and LLaMA-2-7B used different prompts. Bold data is used in shared task training development, and test.
    }
    \label{fig:taskc_data}
\end{figure}

\subsection{Generated text detection}
\label{subsec:subtask_a}

\subsection{Generated text detection: post-deadline additions}
\section{Discussion of SemEval results}
\label{sec:discussion}
\section{Conclusion}
\label{sec:conclusion}

This master thesis has attempted to tackle the problem of machine-generated text detection through the lens of a marked attention to model efficiency and size.
The early chapters served to introduce the field and language modelling in general, hopefully helping to initiate those who were so far unfamiliar with the topic.
Chapter \ref{sec:threats} provided an overview of thread models relating to NLG technologies, such as their potential harmful uses in misinformation campaigns, phishing, or false research.
This was meant as justification for the development of detection technologies: it would be wishful thinking to assume that all actors -- regardless of background, competence, or intention -- will disclose their use of language generation, but alerting users that they're dealing with machine text remains critical.
Following up on the outlined threats, Chapter \ref{sec:approaches} outlined some of the most recent detection approaches present in the scientific literature.
These range from statistical strategies relying on classical representations such as word-frequency vectors and TF-IDF, to solutions employing the same large language models in the detection effort.
Aside from the lively research landscape surrounding the field, especially in recent years, another development that was important to highlight was the movement towards increasingly heavy detection systems.
LLMs and other large transformer-based approaches are responsible for the latest state-of-the-art performance, but these architectures also trade often minor gains for compute requirements that relegate the usage of these solutions to dedicated servers, precluding end users from locally executing software they rely on.

The principal objective of this master thesis is to contribute to the conversation surrounding detection systems by proposing alternative solutions that approximate SOTA performance while maintaining a lean model constitution.
The battlefield of choice was Task 8 at SemEval-2024 \citep{wang2024semeval}, a shared task built around black-box detection of machine generated text.
Model development targeting the shared task took place in roughly two stages: the first while SemEval-2024 was ongoing, resulting in official submissions to the leaderboards under the banner of team TueCICL, and a later phase in the context of this thesis.

The models submitted to the task spanned to subtasks, one consisting in binary classification (subtask A) and the other in change point detection (subtask C).
For both subtasks, the approach was to build an ensemble combining representations from a character-level mode, a TF-IDF model, and a third model build on linguistically motivated features.
This effort resulted in middle-of-the pack rankings, as the submitted solutions fell short of the baseline in both instances, and did not show the convincing performance they exhibited in development.

The second phase of development carried on the work on subtask A, a competition track that challenged participants in pure binary classification over human and machine-generated texts.
Having drawn valuable lesson from the experience during the shared task, and with a more firm grounding the latest research, a new batch of models was developed.
The major introduction in the new stage was to split up the general problem of machine-generated text detection into several sub-problems, aiming to determine whether a target text had been generated by some particular model.
Fifteen single-generator classifiers were trained, three for each of the five models that were included in the data provided for the shared task.
Of the three single-generator classifiers targeting each LLM, one was obtained by fine-tuning DistilBERT, and the other by fitting a random forest classifier with either TF-IDF vectors or linguistically motivated feature representations.
These models were then combined into an ensemble model, which processed each target text by computing the probability that it had been generated by each of the single-generator classifiers, then applying a feed-forward network on the obtained 15-feature vector representation.
This ensemble displayed performance above the task-winning model, and even an ensemble variant that did not have the transformer-based components achieved accuracy levels that would nearly have placed it on the podium.
These results strongly support the argument brought forward by this thesis, i.e., that there exist answers other than huge transformers along the development of high-performing systems.

Team Genaios \citep{sarvazyan-etal-2024-genaios} submitted the system they named LLMixtic, the solution that won the shared.
This model combined token-level probabilistic features extracted with LLaMA-2 models, placing it firmly in the LLM-based category.
Compared to this formulation, the model proposed in this thesis achieves more than comparable performance with only DistilBERT as a transformer-based module, a much smaller model family than Facebook's LLaMA.
Even without DistilBERT, model performance remains high, albeit lower than would be required for the shared task podium.
TF-IDF and especially linguistic features bring another advantage to the table: since the resulting representation is tied to surface-level, conceptually solid aspects, such as the frequency and rarity of a unigrams and bigrams, or some linguistic aspect such as POS-tag distributions, it is possible to trace the final prediction back to these characteristics of the input text.
Even though there weren't any explainability analyses carried out in this work, it is in theory possible to extend the simple binary classification with a natural language reason as to \emph{why} a particular decision was made.

Of course, the comparison with the task-winning model offered by team Genaios is not a very fair one, since test set labels were not available during the shared task.
The model proposed in this work was not developed in the dark, and the possibility to check performance on the test set was used multiple times to determine if an acceptable end result had been reached.
There is a popular saying in the digital space, that hindsight is 20-20, meaning literally that it has perfect eyesight in both eyes.
This phrase describes the follow-up experiments rather aptly: there was a wealth of experience around what had worked and what had not, and a better solution could be built on top of well-understood failures.
This of course only means that the comparison to team Genaios is not exactly correct -- in fact, their achievement, considering the difficulty of the shared task, is more impressive -- but does not take away from the final results that were observed.
On the other hand, the test set was of course not used in training in the post-deadline additions made in this work, making the presented solution at least equally as informative as those submitted during the shared task.

There are also other limitations to the approach outlined in this thesis, which should be mentioned.
One such aspect has to do with the composition of the test set.
Referring back to Table \ref{table:adata}, one can note that there are two aspects that make the test set perhaps not an accurate representation of the potential real-world detection landscape.
First, the only new model introduced by the test set is GTP-4, which has two of its predecessors in the train set (ChatGPT and Davinci-003).
In addition, the test partition contains documents sourced from only one domain: student essays from Outfox \citep{koike2024outfox}.
On the one hand, this puts into question exactly how well do proposed models perform in the black-box context, since it can be argued that the "surprise generator" introduced in the test set may not be very surprising.
On the other, while the Outfox data was not present in either partition other than the test set, it is still only one domain, which makes does not necessarily translate to other generation contexts.
Some domains in which detecting generation may be critical, but that are not present in this dataset, are e-mails (for example related to phishing and scamming attempts), product or place reviews, or general digital content generations, such as blogs or articles, which may be used in disinformation campaigns.

Another limitation of this work is the absence of additional tests concerning adversarial robustness.
Machine-generated text detection is a field particularly characterized by the race between detectors and evaders.
As detection mechanisms become more sophisticated, there comes a growing incentive to develop strategies to evade detection, for example to bypass AI checks that are being rolled out in academic software to ensure that students do not cheat.
Other research \citep{Crothers_2022} has found that statistical features, such as the language metrics included in this study, help the model resist adversarial attacks, since language features are less easily perturbed than LLM embeddings.
The ensemble model presented in this work would benefit from checks to its adversarial robustness, at least to verify if there's a benefit in this area from employing statistical features.

On this note, future research on this model should explore performance on other datasets, as well as in downstream application with real-world users, since it was designed with true deployment in mind.
Another avenue of experimentation could come from updating or adding modules to the ensemble.
Aside from TF-IDF, linguistic features, and DistilBERT, other useful information sources might come from other similar-sized, but differently tuned language models.
Instruction-tuned or chat-based models could provide valuable feedback in more interactive applications, and auxiliary systems augmented with information retrieval, capable of comparing the suspected generation against a corpus of available information, might take the degree of factuality of the target text in consideration when outputting a prediction.
A more competently set-up character-level model approach may also be more informative than the official results in subtask A might suggest -- after all, the ensemble components more closely associated with style, TF-IDF and language features, performed very well in the post-deadline experiments.
There may be more value to be extracted from the characters, especially since they are more commonly paired with convolutional networks in research, whereas they were used alongside an LSTM in this work, an architecture that is not as well-equipped to deal with the long sequence lengths that come with character-level representations.
In addition, the shared task itself contains two interesting directions that were not explored in relation to this ensemble model.

Subtask C of the shared task consisted in change point detection, a detection scenario in which a human author provides the first part of the text, which is then completed by an AI agent.
The use of NLG is harder to detect in this setup, thus making this strategy common in adversarial evasion techniques.
Drawing from the lessons in the simpler binary classification task, more work could be dedicated to change point detection.
Some of the linguistic features do not translate well, which is a challenge, since moving from a global to a per-token or per-subsequence measure is not equally easy for all representations.
Nonetheless, there's little doubt that future contributions to the field cannot afford to simplify their worldview to one in which a text is either fully generated, or fully not.

Alongside the monolingual, English-only dataset, subtask A had a multilingual variant as well.
This parallel subtask was barely mentioned in this work, since it was not undertaken in the official submissions, and the post-deadline additions failed to address it.
However, LLMs are not used only in English -- in fact, cross-linguistic and cross-task performance is one of the evaluation metrics for modern language models.
Smaller models, such as the one proposed in this work, may struggle as the definition of the problem they are solving grows wider -- but this is the exact problem ensembling smaller solutions was meant to address.
As the winning team for the multilingual track, team USTC-BUPT \citep{guo2024ustc} also use a bipartite strategy: after detecting the language of the text, they use a classification head over LLM embeddings for English, and a fine-tuned mT5 variant with a classification token for all other languages.
There is no reason why the specialists that are combined to form the ensemble in this work could not be extended to include classifiers for other languages.
The ways in which such an interplay of modules could be architected are extremely diverse and certainly entice a model designer's creativity.

While more work remains to be done in the field of machine-generated text detection, this master thesis has successfully shown that incremental adoption of LLMs and other similarly huge models is not the only frontier left to explore.
Solutions designed to run locally may be regarded as fringe solutions in the current landscape, but may become critically important in the future.
Research in this direction may not produce the highest-performing state-of-the-art systems, but it will certainly cover the many use cases in which detection software cannot be offered as a remote service.
Though different researchers and developers are bound to have different views in what they regard as interesting and innovative, there's also a point to be made in favor of the creativity inherent in making the most of whatever resources are available, as opposed to picking the latest off-the-shelf transformer model and fine-tuning it on the latest data collection.
Putting together a system that may be odd and wonky at the start and taking it off the ground offers a sense of achievement if it succeeds and a wealth of lessons when it fails, only to the benefit of end users.
\input{sections/08-acknowledgements.tex}

% Bibliography entries for the entire Anthology, followed by custom entries
%\bibliography{anthology,custom}
% Custom bibliography entries only
\bibliography{anthology,custom,semeval}

\end{document}
